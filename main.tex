\documentclass[12pt,article,oneside,a4paper]{abntex2}

% NESSA SEÇÃO VOCÊ PODE ADICIONAR OS PACOTES PARA EDIÇÃO DO SEU TEXTO
\usepackage[brazil]{babel}
\usepackage[utf8]{inputenc}
\usepackage[T1]{fontenc}
\usepackage{lipsum} 
\usepackage{graphicx} 
\usepackage{url}
\usepackage{newtxtext} 
\usepackage{changepage}
\usepackage{setspace}
\usepackage{natbib}
\usepackage{ulem}

% INICIO DO DOCUMENTO
\begin{document}

% INFORMAÇÕES DO CABEÇALHO
{\centering
\includegraphics[height=2.46cm, width=3.47cm]{image1.png}\\[1em] 
{\fontfamily{qtm}\selectfont\MakeUppercase{\textbf{Centro Universitário de Lavras}}\\[0.1em]}

% AQUI VOCE PODE ADICIONAR O <TITULO DO SEU TRABALHO>
{\fontfamily{qtm}\selectfont\MakeUppercase{Nome da Atividade}\\[0.1em]}
}
% AQUI VOCE PODE ADICIONAR O SEU <NOME>
{\flushright
{\fontfamily{qtm}\selectfont{Nome do aluno}}\\[1em]
}

% CONFIGURAÇÕES DO PARAGRAFO
\setlength{\parindent}{1.25cm} % Recuo da primeira linha do parágrafo
\setlength{\parskip}{0pt} % Espaçamento entre parágrafos

% AQUI VOCE PODE ADICIONAR O SEU TEXTO

% ESSA OBSERVAÇÃO PODE SER APAGADA OU DESABILITADA COLOCANDO O SINAL DE % ANTES DE CADA LINHA COMO NO EXEMPLO ABAIXO, ELE SERVE PARA FAZER UM COMENTARIO QUE NÃO SERÁ IMPRESSO OU DESABILITAR UM CONJUNTO DE LINHAS:
%
%
%\begin{center}
%\textbf{\uline{Observação: O texto deve ser escrito em Arial ou Times New Roman, tamanho 12.}}
%\end{center}
%

\begin{center}
\textbf{\uline{Observação: O texto deve ser escrito em Arial ou Times New Roman, tamanho 12.\newline}}
\end{center}

% section{Introdução} - ADICIONA O CAPITULO NUMERAO
% section*{Introdução} - ADICIONA O CAPITULO SEM NUMERAÇÃO


Este texto foi escrito para ser utilizado como base de formatação de seu trabalho. À medida que for redigindo o documento, apague-o.
O Manual de Normas para Elaboração de Trabalhos Acadêmicos - UNILAVRAS e a ABNT - Associação Brasileira de Normas Técnicas sugerem que, ao escrever o texto, deve-se utilizar um recuo de 1,25 cm na primeira linha do parágrafo e que entre um parágrafo e outro, não deve haver espaçamento. Seguindo a formatação deste documento você já está cumprindo tais normas.

Ao redigir um texto, é importante que você busque suporte teórico e fontes de pesquisa que embasem sua escrita, conferindo-a credibilidade. A citação, pertinente, de tais fontes se faz necessária para que não se caracterize o plágio, que é a "apropriação indevida de um texto ou parte dele, sem referência ao autor, portanto apresentado como sendo de autoria da pessoa que dele se apodera." (SILVA, 2008, p. 360). O trecho anterior, é um tipo de citação: a citação direta curta. Como você pôde perceber, ela se atém a três linhas, e por isso veio incorporada ao parágrafo, entre aspas duplas, seguida pela indicação do autor, ano e página da publicação. 
A citação direta pode ser curta, como já mencionado e demonstrado acima, ou longa, conforme o exemplo abaixo.

% MODELO DE CITAÇÃO DIRETA:

\begin{adjustwidth}{5.5cm}{0cm}
\fontsize{10pt}{1.2em}\selectfont 
\noindent uma cópia literal do texto. Transcrevem-se geralmente decretos, regulamentos, leis, fórmulas científicas ou trechos de obras. O tamanho da citação determinará sua localização no trabalho. Se a citação tiver até três linhas, virá incorporada ao parágrafo, entre aspas duplas. As citações com mais de três linhas ficarão abaixo do parágrafo, em bloco, com início sob a linha anterior, a quatro cm à direita da tabulação, em espaço simples”. \citealp{oliveira2018}\newline 
\end{adjustwidth}

O trecho recuado acima, é um exemplo de citação direta longa. Este possui mais de três linhas e por isso, está a 4 cm da margem esquerda, com fonte 10, sem espaçamento e aspas e com menção à autora (sobrenome em caixa alta), ano e página. 

Além das citações diretas, na escrita do texto, podemos usufruir da citação indireta que é advinda da interpretação das ideias lançadas pelo autor/a do documento \citep{abnt2022}. Um exemplo deste tipo de citação foi dado, no início do presente parágrafo, ao descrever, sem transcrever, o que sugere o documento da ABNT sobre o conceito de citação indireta. 

Uma outra forma de utilizar tal citação, é contextualizando, como sugere Oliveira (2018), a linha de raciocínio do/a autor/a com a sua interpretação no decorrer do texto. Percebeu como? 
Importante destacar que para os três tipos de citação, se faz interessante e pertinente moldá-las e contextualizá-las, tornando-as parte corrente do texto e não, simplesmente, colocá-las no corpo deste.

Portanto, é de suma importância descrever com suas palavras o trecho pesquisado em livros, periódicos, unidades do SAGAH, plataformas digitais de pesquisa acadêmica (SciELO, ERIC, Periódicos CAPES e Google Acadêmico), citando a referência. Está é a forma correta de redigir durante a redação de um trabalho acadêmico.

Além das citações, é importante que se referencie corretamente a(s) fonte(s) de suas pesquisas, organizando-as em ordem alfabética, deixando-as alinhadas à esquerda, com espaçamento entrelinhas simples e separadas entre si com 1,5 \citep{abnt2022}. 

Veja a sequência do que deve conter na referência e os exemplos de cada caso:\newline

\noindent \textbf{\textit{Livros: SOBRENOME, Iniciais do(s) primeiro(s) nome(s). Nome do livro. Edição. Lugar da publicação: Editora, ano da publicação.}}\newline

\noindent Artigo científico: SOBRENOME, Iniciais do(s) primeiro(s) nome(s). Nome do artigo. Nome
da revista em que foi publicado, local em que foi publicado, volume, número, intervalo de
páginas, data da publicação\newline

\noindent \textbf{\textit{Artigo científico: SOBRENOME, Iniciais do(s) primeiro(s) nome(s). Nome do artigo. Nome da revista em que foi publicado, local em que foi publicado, volume, número, intervalo de páginas, data da publicação.}}\newline

\noindent \textbf{ OSTINI, F.M.et al. O uso de drogas vasoativas em terapia intensiva. Medicina – Revista do Hospital das Clínicas e da Faculdade de Medicina de Ribeirão Preto da Universidade de São Paulo, Ribeirão Preto, v.31, n23, p. 400-411, jul/set. 1998.}\newline

\noindent \textbf{\textit{Documento eletrônico: RESPONSÁVEL PELA PUBLICAÇÃO. Nome do documento. Local da publicação, data da publicação. Disponível em Link do documento, data de acesso.}} 

\noindent CONSELHO NACIONAL DE ÉTICA PARA CIÊNCIA DA VIDA. Reflexão ética sobre a dignidade humana. Lisboa, 5 jan. 1999. Disponível em http://www.cnecv.gov.pt/pdfs/dighum.pdf. Acesso em 26 set. 2000\newline         

É importante buscar embasamento teórico em sites e plataformas que possuam credibilidade. Sendo assim, sugerimos, por exemplo, SciELO, ERIC, Periódicos CAPES, Google Acadêmico, além do ProQuest disponível no portal acadêmico do UNILAVRAS. 

Esperamos ter tornado o processo de escrita menos árduo com essas dicas. Utilize-as e coloque em prática a escrita acadêmica, em seus desafios e demais atividades realizadas no decorrer de seu curso.\newline

Bom trabalho!!


% REFERÊNCIAS
\bibliographystyle{plainnat}
\bibliography{referencias}

%FIM DO DOCUMENTO
\end{document}
